%=======================02-713 LaTeX template, following the 15-210 template==================
%
% You don't need to use LaTeX or this template, but you must turn your homework in as
% a typeset PDF somehow.
%
% How to use:
%    1. Update your information in section "A" below
%    2. Write your answers in section "B" below. Precede answers for all 
%       parts of a question with the command "\question{n}{desc}" where n is
%       the question number and "desc" is a short, one-line description of 
%       the problem. There is no need to restate the problem.
%    3. If a question has multiple parts, precede the answer to part x with the
%       command "\part{x}".
%    4. If a problem asks you to design an algorithm, use the commands
%       \algorithm, \correctness, \runtime to precede your discussion of the 
%       description of the algorithm, its correctness, and its running time, respectively.
%    5. You can include graphics by using the command \includegraphics{FILENAME}
%
\documentclass[11pt]{article}
\usepackage{amsmath,amssymb,amsthm}
\usepackage{graphicx}
\usepackage[margin=1in]{geometry}
\usepackage{fancyhdr}
\usepackage{caption}
\setlength{\parindent}{0pt}
\setlength{\parskip}{5pt plus 1pt}
\setlength{\headheight}{13.6pt}
\newcommand\question[2]{\vspace{.25in}\hrule\textbf{#1: #2}\vspace{.5em}\hrule\vspace{.10in}}
\renewcommand\part[1]{\vspace{.10in}\textbf{(#1)}}
\newcommand\algorithm{\vspace{.10in}\textbf{Algorithm: }}
\newcommand\correctness{\vspace{.10in}\textbf{Correctness: }}
\newcommand\runtime{\vspace{.10in}\textbf{Running time: }}
\pagestyle{fancyplain}
\lhead{\textbf{\NAME\ }}
\chead{\textbf{Lab \HWNUM}}
\rhead{ \today}
\begin{document}\raggedright
%Section A==============Change the values below to match your information==================
\newcommand\NAME{[CS333] Operating Systems Lab}  % your name
\newcommand\ANDREWID{}     % your andrew id
\newcommand\HWNUM{04}              % the homework number
%Section B==============Put your answers to the questions below here=======================

% no need to restate the problem --- the graders know which problem is which,
% but replacing "The First Problem" with a short phrase will help you remember
% which problem this is when you read over your homeworks to study.

\begin{center}
\center \textbf{Submitted by:}
\end{center}
\begin{table}[htb]
\centering
\begin{tabular}{ l r }
 \hline
 \textbf{Name} & \textbf{Roll No.} \\
 \hline
 Rupanshu Ganvir & 140050005 \\ 
 Utkarsh Gautam & 140050009 \\
 \hline
\end{tabular}
\end{table}


\question{1}{Queue size unbounded} 

We ran the experiment twice.

The maximum throughput is achieved at N=3 or N=4.

This is the number of worker threads that saturate the server.


\question{2}{Queue size of one}

The throughput keeps on increasing and doesn't stabilise. 

Yes the clients are denied service in some runs and the connection times out.

In our implementation in such a case, the client thread just continues 
the next iteration of while loop.

Also, the throughput is higher than that in Ex.1. 

This can be explained as follows: When queue size is unbounded, the requests 
are all \texttt{accept}ed by the server, but may not be served until some worker thread
is ready to handle it. The thread was then context switched out.
It might take some time for this thread to become active again.
Lets say the client thread thus exists in a limbo state.

On the other hand, in this exercise, since queue size is bounded, there can 
only be a limited number of client thread in limbo state(ones that have been 
\texttt{accept}ed but not served). Hence, the throughput is expected to be higher. 


\flushright{* experimental data can be dound in \texttt{readme.txt}}

\end{document}
